\chapter*{Liste des abréviations}
\addcontentsline{toc}{chapter}{Liste des abréviations}

\begin{longtable}{|p{3cm}|p{11cm}|}
\hline
\textbf{Abréviation} & \textbf{Définition} \\
\hline
CNN & Convolutional Neural Network (réseau de neurones convolutifs) \\
\hline
TBNet & \textit{Tiny Bean (cooking time) Network}, architecture proposée pour la prédiction du temps de cuisson des haricots à partir d’images. 
Le suffixe \textbf{2} (TBNet2) correspond à une couche finale de 256 filtres, tandis que le suffixe \textbf{5} (TBNet5) correspond à 512 filtres. \\
\hline
MAE & Mean Absolute Error (erreur absolue moyenne), mesure de l’écart moyen entre les valeurs prédites et observées. \\
\hline
RMSE & Root Mean Square Error (racine de l’erreur quadratique moyenne), mesure de l’écart quadratique entre prédictions et observations. \\
\hline
$R^2$ & Coefficient de détermination, mesure de la proportion de variance expliquée par le modèle. \\
\hline
MAPE & Mean Absolute Percentage Error (erreur absolue moyenne en pourcentage). \\
\hline
MaxErr & Maximum Error (erreur maximale observée). \\
\hline
TinyML & Tiny Machine Learning, déploiement de modèles d’apprentissage automatique sur des dispositifs embarqués à faibles ressources. \\
\hline
XAI & Explainable Artificial Intelligence (intelligence artificielle explicable), ensemble de méthodes visant à interpréter les décisions des modèles. \\
\hline
\end{longtable}
