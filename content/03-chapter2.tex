\chapter{Développement et implémentation}
\label{chap:developpement}

Ce chapitre détaille la mise en œuvre concrète du système de prédiction du temps de cuisson des haricots, depuis la préparation des données jusqu’au déploiement embarqué. Il s’appuie sur la méthodologie définie au Chapitre~\ref{chap:methodologie} et en reprend strictement les choix techniques (prétraitements, architectures candidates, schéma d’entraînement et optimisation TinyML).

\section{Environnement de développement}
\label{sec:env_dev}

\subsection{Matériel}
Les expérimentations ont été menées sur :
\begin{itemize}
    \item \textbf{Machine locale} : Intel Core i7 (4 cœurs), 16~Go RAM.
    \item \textbf{Google Colab} : session `Testa 4` (15~Go VRAM, 12~Go RAM, 118~Go stockage).
\end{itemize}

\subsection{Logiciels et versions}
Sauf mention contraire, les versions sont celles déjà fixées (voir Chap.~\ref{chap:developpement} version antérieure) :
\begin{itemize}
    \item \textbf{Python} 3.10
    \item \textbf{TensorFlow} 2.13 \& API Keras (compatibles \texttt{TensorFlow Lite})
    \item \textbf{Pandas} 2.1, \textbf{NumPy} 1.25, \textbf{Matplotlib} 3.8
    \item (le cas échéant, outils d’analyse complémentaires cohérents avec la méthodologie)
\end{itemize}

\paragraph{Placeholders installation/env.}
\begin{lstlisting}[language=bash,caption={(Placeholder) Installation/gel des dépendances},label={lst:env_install}]
# >>> CODE ICI <<<
# Exemple:
# pip install tensorflow==2.13.0 pandas==2.1.* numpy==1.25.* matplotlib==3.8.*
# pip freeze > requirements.txt
\end{lstlisting}

\section{Organisation des sources et données}
\label{sec:orga_sources}

\paragraph{Arborescence (placeholder).}
\begin{lstlisting}[language=bash,caption={(Placeholder) Arborescence du projet},label={lst:tree}]
# >>> CODE ICI <<<
# Exemple (à adapter):
# project/
# ├─ data/
# │  ├─ raw/           # images brutes
# │  ├─ splits/        # index train/val/test
# │  └─ hdf5/          # jeux HDF5 préparés
# ├─ src/
# │  ├─ preprocess/    # scripts prétraitement/offline
# │  ├─ training/      # scripts entrainement
# │  ├─ export/        # conversion TFLite
# │  └─ android/       # app Android (Kotlin)
# └─ reports/          # logs TensorBoard, figures, tableaux
\end{lstlisting}

\section{Préparation des données}
\label{sec:pretraitement}

Le protocole de préparation suit \textit{strictement} le Chapitre~\ref{chap:methodologie} :
\begin{enumerate}
    \item \textbf{Recadrage centré} et \textbf{redimensionnement} des images brutes (\(3000\times4000\)) en \(224\times224\).
    \item \textbf{Mise à l’échelle} des pixels dans \([0,1]\).
    \item \textbf{Augmentation} stochastique \textit{train-only} (flips, crops, jitter de luminosité/contraste/saturation, blur, bruit gaussien)\footnote{Paramètres détaillés et justification : voir Chap.~\ref{chap:methodologie}.}.
    \item \textbf{Normalisation Min--Max} des labels \(T_c\in[0,1]\) calculée \emph{uniquement} sur \(\mathcal{D}_{\text{train}}\) et réappliquée à val/test; inversion pour revenir aux minutes en sortie.
    \item \textbf{Export HDF5} par split avec clés \texttt{images}, \texttt{cooking\_times}.
\end{enumerate}

\paragraph{Scripts de préparation (placeholders).}
\begin{lstlisting}[language=Python,caption={(Placeholder) Génération des splits et export HDF5},label={lst:prep_hdf5}]
# >>> CODE ICI <<<
# - chargement des chemins images + labels (CSV/JSON)
# - génération des index train/val/test (seed fixée)
# - pipeline: crop/resize 224x224 + normalisation [0,1]
# - augmentation (train uniquement)
# - scaling Min-Max de Tc avec Tc_min/Tc_max de train
# - écriture HDF5: images (uint8/float16), labels (float32), meta
\end{lstlisting}

\paragraph{Chargement par lot (mémoire contrainte).}
\begin{lstlisting}[language=Python,caption={(Placeholder) DataLoader/TFDataset depuis HDF5},label={lst:dataloader}]
# >>> CODE ICI <<<
# - ouverture HDF5 (mode lecture)
# - dataset tf.data / generator Python
# - batching, shuffle (train), prefetch
\end{lstlisting}

\section{Architectures de modèles}
\label{sec:modeles}

Conformément à la méthodologie, nous considérons :
\begin{itemize}
    \item \textbf{CNN personnalisés} (deux variantes) pour extraction hiérarchique efficace.
    \item \textbf{MobileNetV2}, \textbf{EfficientNetB0}, \textbf{NASNetMobile} (apprentissage par transfert).
\end{itemize}
La tête de régression utilise une sortie scalaire avec activation \texttt{linear}. La régularisation reprend les choix de base (dropout 0{,}3, pénalisation L2 \(\lambda=10^{-4}\)).

\paragraph{CNN personnalisé (placeholder).}
\begin{lstlisting}[language=Python,caption={(Placeholder) Définition d'un CNN personnalisé (Keras)},label={lst:cnn_custom}]
# >>> CODE ICI <<<
# def build_cnn_custom(input_shape=(224,224,3), l2_reg=1e-4, dropout=0.3):
#     ...
#     return model
\end{lstlisting}

\paragraph{Modèles pré-entraînés (placeholder).}
\begin{lstlisting}[language=Python,caption={(Placeholder) Transfer learning MobileNetV2/EfficientNetB0/NASNetMobile},label={lst:transfer}]
# >>> CODE ICI <<<
# - chargement du backbone (weights="imagenet", include_top=False, input_shape=224x224x3)
# - global average pooling
# - dense(s) + régularisation (L2, dropout)
# - sortie Dense(1, activation="linear")
# - stratégie de gel/dégel des couches (fine-tuning)
\end{lstlisting}

\section{Stratégie d’entraînement}
\label{sec:entrainement}

Les hyperparamètres de base sont ceux fixés précédemment (voir Chap.~\ref{chap:methodologie}) :
\begin{itemize}
    \item Optimiseur \textbf{Adam} (\(\beta_1=0{,}9\), \(\beta_2=0{,}999\)), LR initial \(10^{-4}\), \texttt{ReduceLROnPlateau}.
    \item Perte \textbf{MSE}; métriques surveillées : MAE, RMSE, \(R^2\) (et MAPE si requis).
    \item \textbf{Batch size}~32; \textbf{max}~100 époques; \textbf{early stopping} (patience 5).
    \item Régularisation : \textbf{dropout 0{,}3}, \textbf{L2} \(10^{-4}\).
\end{itemize}

\paragraph{Compilation/entraînement (placeholders).}
\begin{lstlisting}[language=Python,caption={(Placeholder) Compilation et fit Keras},label={lst:fit}]
# >>> CODE ICI <<<
# model.compile(optimizer=..., loss="mse", metrics=[...])
# callbacks = [EarlyStopping(...), ReduceLROnPlateau(...), ModelCheckpoint(...)]
# history = model.fit(train_ds, validation_data=val_ds, epochs=100, callbacks=callbacks)
\end{lstlisting}

\paragraph{Journalisation et réplication.}
\begin{lstlisting}[language=Python,caption={(Placeholder) Seeds, TensorBoard, sauvegarde checkpoints},label={lst:tb}]
# >>> CODE ICI <<<
# - fix seeds (python/numpy/tf)
# - TensorBoard (logs dir)
# - sauvegarde des checkpoints .keras/.h5
\end{lstlisting}

\section{Export et optimisation TinyML}
\label{sec:optim_tinyml}

\subsection{Conversion TensorFlow Lite (float16)}
Conformément à la décision validée, la quantification retenue est \textbf{float16} (réduction mémoire, bonne fidélité). 

\paragraph{Conversion (placeholder).}
\begin{lstlisting}[language=Python,caption={(Placeholder) Conversion TFLite en float16},label={lst:tflite_fp16}]
# >>> CODE ICI <<<
# converter = tf.lite.TLiteConverter.from_keras_model(model)
# converter.optimizations = [tf.lite.Optimize.DEFAULT]
# converter.target_spec.supported_types = [tf.float16]
# tflite_model = converter.convert()
# with open("model_fp16.tflite","wb") as f: f.write(tflite_model)
\end{lstlisting}

\subsection{Validation TFLite vs. modèle de référence}
\paragraph{Parité de sortie (placeholder).}
\begin{lstlisting}[language=Python,caption={(Placeholder) Test de parité Keras vs TFLite sur N échantillons},label={lst:parity}]
# >>> CODE ICI <<<
# - charger "model_fp16.tflite" dans un Interpreter
# - comparer sorties (après inverse Min-Max) vs prédictions Keras
# - rapporter MAE/RMSE de parité (à insérer dans le chap. Résultats)
\end{lstlisting}

\section{Déploiement Android}
\label{sec:deploiement_android}

\subsection{Cible et outils}
\begin{itemize}
    \item \textbf{Android Studio} Narwhal (2025.1.2), \textbf{SDK} Android \textbf{36}.
    \item Compatibilité descendante \texttt{minSdkVersion=24} (Android 7.0).
    \item \textbf{Langage} : Kotlin.
    \item \textbf{Runtime ML} : \texttt{TensorFlow Lite Interpreter} v2.13.
\end{itemize}

\subsection{Intégration et inférence}
Le modèle \texttt{.tflite} est placé dans \texttt{assets/}. Le pipeline embarqué reproduit le prétraitement (redimensionnement \(224\times224\), normalisation cohérente), exécute l’inférence, puis applique l’inversion Min--Max pour restituer \(T_c\) en minutes.

\paragraph{Gradle (placeholder).}
\begin{lstlisting}[language=groovy,caption={(Placeholder) Dépendances Gradle TFLite},label={lst:gradle}]
# >>> CODE ICI <<<
# implementation 'org.tensorflow:tensorflow-lite:2.13.0'
# (ajouter delegates NNAPI/GPU si nécessaire)
\end{lstlisting}

\paragraph{Kotlin: chargement modèle (placeholder).}
\begin{lstlisting}[language=Java,caption={(Placeholder) Kotlin - chargement et exécution TFLite},label={lst:kotlin_tflite}]
/* >>> CODE ICI <<<
// fun loadModel(context): MappedByteBuffer { ... }
// val interpreter = Interpreter(loadModelFile("model_fp16.tflite"))
// val inputBuffer: ByteBuffer = preprocessToInputBuffer(bitmap224x224) // normalisation cohérente
// interpreter.run(inputBuffer, outputBuffer)
// val y_pred_minutes = inverseMinMax(outputBuffer, Tc_min, Tc_max)
*/
\end{lstlisting}

\paragraph{UI/Workflow (placeholder figure).}
\begin{figure}[h!]
    \centering
    % >>> FIGURE ICI <<< (schéma UI: capture -> prétraitement -> inférence -> restitution Tc)
    \caption{Workflow fonctionnel de l’application Android (capture, prétraitement, inférence, affichage).}
    \label{fig:app_workflow}
\end{figure}

\section{Mesures et artefacts à produire}
\label{sec:mesures_artefacts}

Cette section ne présente \emph{pas} de résultats chiffrés (rapportés au Chap. Résultats), mais liste les artefacts à générer pour chaque modèle testé.

\paragraph{Taille binaire et compatibilité.}
\begin{lstlisting}[language=bash,caption={(Placeholder) Rapport de taille et compatibilité modèles},label={lst:size_report}]
# >>> CODE ICI <<<
# - taille du fichier .tflite (float32 vs float16)
# - (optionnel) signature TFLite, infos d'entrées/sorties
\end{lstlisting}

\paragraph{Tableau comparatif (placeholder valeurs).}
\begin{table}[h!]
\centering
\caption{(Placeholder) Synthèse des modèles déployables (à remplir avec valeurs mesurées).}
\label{tab:deploy_synthese}
\begin{tabular}{@{}lcccc@{}}
\toprule
Modèle & Fichier TFLite & Taille (Mo) & Latence (ms)\textsuperscript{$\dagger$} & Notes \\ \midrule
CNN personnalisé & model\_fp16.tflite & \textit{à insérer} & \textit{à insérer} & float16 \\
MobileNetV2      & model\_fp16.tflite & \textit{à insérer} & \textit{à insérer} & float16 \\
EfficientNetB0   & model\_fp16.tflite & \textit{à insérer} & \textit{à insérer} & float16 \\
NASNetMobile     & model\_fp16.tflite & \textit{à insérer} & \textit{à insérer} & float16 \\ \bottomrule
\end{tabular}

\begin{flushleft}
\footnotesize{$\dagger$ Mesures \emph{sur appareil cible} : moyenne et percentiles; reportées au Chapitre Résultats.}
\end{flushleft}
\end{table}

\section{Reproductibilité et CI (placeholders)}
\label{sec:ci_repro}

\paragraph{Scripts et seeds.}
\begin{lstlisting}[language=bash,caption={(Placeholder) Exécution bout-en-bout},label={lst:run_all}]
# >>> CODE ICI <<<
# 1) python src/preprocess/make_hdf5.py --config ...
# 2) python src/training/train.py --model cnn_custom --epochs 100 ...
# 3) python src/export/to_tflite.py --quant float16 ...
# 4) python src/validate/parity_tflite.py --n 256 ...
\end{lstlisting}

\paragraph{Tests de non-régression.}
\begin{lstlisting}[language=Python,caption={(Placeholder) Tests unitaires/parité sortie},label={lst:tests}]
# >>> CODE ICI <<<
# - test: même normalisation/offline vs online
# - test: parité Keras vs TFLite (tolérance MAE)
\end{lstlisting}

\section{Limites pratiques et points d’attention}
\label{sec:limites}
\begin{itemize}
    \item \textbf{Alignement des prétraitements} : le pipeline embarqué doit reproduire exactement la normalisation (y compris l’inversion Min--Max).
    \item \textbf{Latence et consommation} : les valeurs dépendent fortement de l’appareil et des délégués (CPU/GPU/NNAPI); les mesures sont reportées au Chapitre Résultats.
    \item \textbf{Gestion mémoire} : réutilisation des buffers d’E/S et chargement paresseux recommandés côté Android.
\end{itemize}

\bigskip
\noindent \textbf{Résumé} — Ce chapitre fournit la structure d’implémentation complète (prétraitement, modèles, entraînement, conversion float16, intégration Android) et des \textit{placeholders} pour insérer le code et les mesures réelles sans introduire d’informations non vérifiées.
Il sert de guide pour la réalisation pratique et la documentation des résultats, qui seront présentés et analysés en détail au Chapitre~\ref{chap:resultats_discussion}.