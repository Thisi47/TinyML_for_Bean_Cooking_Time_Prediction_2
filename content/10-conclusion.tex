\chapter{Conclusion et perspectives}
\label{chap:conclusion_et_perspectives}

\section{Synthèse des contributions}
\label{sec:synthese_contributions}

Ce mémoire a exploré l’application du \textit{Tiny Machine Learning} (TinyML) à la prédiction du temps de cuisson des haricots à partir d’images. Plusieurs contributions scientifiques et techniques peuvent être soulignées :

\begin{enumerate}
    \item \textbf{Constitution d’un jeu de données original.} Un corpus inédit a été établi, comprenant des images annotées de 56 variétés de haricots, associées à des temps de cuisson mesurés de manière expérimentale. Cette base constitue un apport méthodologique utile pour la recherche en agroalimentaire et en vision par ordinateur.
    
    \item \textbf{Prétraitement et normalisation.} Un protocole rigoureux a été défini pour préparer les données : redimensionnement homogène des images, augmentation artificielle pour compenser les déséquilibres inter-variétés, et normalisation afin de stabiliser l’entraînement.
    
    \item \textbf{Conception de modèles adaptés au TinyML.} Plusieurs architectures ont été testées, incluant des réseaux pré-entraînés (MobileNetV2, EfficientNetB0, NASNetMobile) et des CNN compacts développés sur mesure. Les expérimentations ont démontré que les modèles personnalisés offrent le meilleur compromis entre précision prédictive et contraintes de calcul.
    
    \item \textbf{Évaluation approfondie.} Les performances ont été analysées à travers plusieurs indicateurs (MAE, RMSE, $R^2$, MAPE), permettant de mettre en évidence une corrélation satisfaisante entre valeurs prédites et observées, validant ainsi la pertinence de l’approche.
    
    \item \textbf{Quantification et portabilité.} Les modèles ont été compressés et convertis en formats adaptés au déploiement embarqué, ouvrant la voie à une utilisation sur des dispositifs contraints tels que les microcontrôleurs ARM Cortex-M.
\end{enumerate}

\section{Bilan critique}
\label{sec:bilan_critique}

\subsection{Points forts}
\begin{itemize}
    \item \textbf{Valeur ajoutée du jeu de données.} La constitution d’un jeu de données spécialisé, inédit dans le contexte local, renforce l’originalité et l’intérêt scientifique du travail.
    \item \textbf{Orientation vers le TinyML.} L’adaptation des modèles aux contraintes du calcul embarqué montre une approche pragmatique, tournée vers l’application réelle.
    \item \textbf{Méthodologie robuste.} L’évaluation croisée par plusieurs métriques et modèles confère de la solidité aux conclusions.
\end{itemize}

\subsection{Limites}
\begin{itemize}
    \item \textbf{Taille et diversité du jeu de données.} La base d’images, bien que précieuse, reste de taille limitée et ne couvre pas toute la variabilité possible des haricots.
    \item \textbf{Sensibilité des mesures expérimentales.} Le temps de cuisson dépend de nombreux facteurs (qualité de l’eau, dureté initiale des grains, conditions environnementales), introduisant un bruit difficilement contrôlable. %\cite{luthra2018cooking}.
    \item \textbf{Prédiction unidimensionnelle.} Le système repose uniquement sur des images statiques, sans intégration d’autres variables complémentaires comme la composition chimique ou l’humidité.
\end{itemize}

\section{Perspectives}
\label{sec:perspectives}

Les résultats obtenus ouvrent plusieurs pistes de recherche et d’application :

\begin{enumerate}
    \item \textbf{Extension du jeu de données.} L’élargissement à de nouvelles variétés et à des conditions expérimentales diversifiées améliorerait la robustesse et la généralisabilité du modèle.
    
    \item \textbf{Fusion multimodale.} L’intégration de données complémentaires (spectroscopie, mesures physico-chimiques, teneur en eau) permettrait de réduire l’incertitude des prédictions et d’augmenter la précision.
    
    \item \textbf{Optimisation avancée pour TinyML.} L’exploration de techniques telles que la distillation de connaissances \cite{hinton2015distillation}, la quantification dynamique \cite{jacob2018quantization} ou le \textit{pruning} \cite{han2015deep} constituerait une étape supplémentaire vers des modèles plus légers et efficaces.
    
    \item \textbf{Validation en conditions réelles.} Le déploiement sur prototypes embarqués dans un contexte agroalimentaire permettrait d’évaluer concrètement la fiabilité du système et d’identifier les besoins d’adaptation industrielle.
    
    \item \textbf{Explicabilité des modèles.} L’utilisation d’approches d’IA explicables (\textit{Explainable AI}, XAI) offrirait une meilleure compréhension des caractéristiques visuelles exploitées par le réseau \cite{arrieta2020explainable}, renforçant ainsi l’acceptabilité et la confiance des utilisateurs finaux.
\end{enumerate}

\section{Conclusion générale}
\label{sec:conclusion_generale}

En définitive, ce mémoire met en évidence la pertinence d’appliquer des approches de vision par ordinateur et d’apprentissage profond à une problématique agroalimentaire concrète : l’estimation du temps de cuisson des haricots. 

En combinant constitution de données originales, développement de modèles adaptés au TinyML, et évaluation méthodique, cette recherche illustre comment l’intelligence artificielle embarquée peut être mobilisée au service de l’agriculture et de l’industrie alimentaire. 

Les limites identifiées et les perspectives proposées ouvrent la voie à des travaux futurs visant à accroître la robustesse, l’explicabilité et l’applicabilité du système. Ainsi, ce travail constitue une étape significative vers l’intégration de solutions d’IA embarquée dans la transformation et la valorisation des produits agricoles.
