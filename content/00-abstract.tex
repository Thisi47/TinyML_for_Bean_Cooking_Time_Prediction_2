\chapter*{Abstract}

The optimization of food processing is crucial to improve product quality, reduce costs, and minimize waste. Among these processes, bean cooking time is decisive, as it directly affects texture, nutritional value, and consumer acceptance. Traditional estimation methods, although reliable, are time-consuming, destructive, and unsuitable for online control. This thesis introduces an innovative approach based on \emph{Tiny Machine Learning} (TinyML) to predict bean cooking time from images. A hybrid pipeline combining classification and regression was designed, leveraging lightweight neural architectures (custom CNNs, MobileNetV2, EfficientNet-B0, NASNetMobile) optimized and quantized for embedded deployment. Results highlight that the custom CNN achieves the best balance between accuracy and efficiency. Integrated into an Android application, the system demonstrates the feasibility of fast and reliable on-device prediction. Beyond technical contributions, this work illustrates the potential of TinyML in the agri-food sector and points to new perspectives for sustainable quality control.

\vspace{0.5cm}
\noindent\textbf{Keywords:} TinyML, computer vision, lightweight neural networks, beans, cooking time.