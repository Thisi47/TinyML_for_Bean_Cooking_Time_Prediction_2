\chapter{Introduction générale}
\label{chap:introduction_generale}

\section{Contexte et justification}
Les projections démographiques annoncent une croissance continue de la population mondiale, qui devrait atteindre près de 9,7 milliards d’habitants à l’horizon 2050 \cite{UN_WPP_2022}. Cette évolution impose des défis sans précédent en matière de sécurité alimentaire, de gestion durable des ressources et de résilience des systèmes de production. Dans ce contexte, l’optimisation des procédés de transformation agroalimentaire apparaît comme un enjeu stratégique majeur. Elle conditionne non seulement la qualité et la sécurité des produits, mais aussi la maîtrise des coûts de production et la réduction des pertes post-récolte, dans un monde où la pression sur les ressources naturelles ne cesse de croître.

Parmi ces procédés, la cuisson des légumineuses — et plus particulièrement des haricots — occupe une place essentielle. Représentant une source majeure de protéines, de fibres et de micronutriments, les haricots jouent un rôle central dans l’équilibre nutritionnel et la souveraineté alimentaire de nombreuses régions du globe \cite{mendoza2018prediction,mbofung2012proximate}. Leur accessibilité économique en fait une denrée privilégiée pour répondre aux besoins nutritionnels des populations croissantes, notamment dans les pays en développement. Toutefois, le temps de cuisson des haricots constitue un facteur critique : il influence directement la texture, la saveur et la valeur nutritionnelle, tout en déterminant leur acceptabilité par les consommateurs.

Les méthodes conventionnelles d’estimation du temps de cuisson, basées sur des approches empiriques ou destructives (tests de cuisson, mesures d’absorption d’eau), bien que fiables, se révèlent chronophages, coûteuses et difficilement adaptables à une production industrielle en flux tendu. Elles entraînent également une consommation énergétique significative, peu compatible avec les impératifs actuels de durabilité. Ces contraintes affectent particulièrement les petites et moyennes unités de transformation, souvent dépourvues d’équipements sophistiqués et confrontées à des marges de manœuvre limitées.

L’émergence de l’intelligence artificielle (IA) et des technologies embarquées ouvre de nouvelles perspectives pour dépasser ces limitations. Le \textit{Tiny Machine Learning} (TinyML), en particulier, permet de déployer des modèles légers directement sur des dispositifs à faibles ressources, tels que les microcontrôleurs ou les smartphones, sans nécessiter d’infrastructure lourde ni de connexion internet permanente \cite{banbury2021micronets}. Grâce à des architectures compactes comme \textit{MobileNetV2} ou des réseaux convolutionnels sur mesure, il devient possible de traiter des données visuelles localement, avec un coût énergétique réduit. Cette approche rend l’IA accessible dans des environnements contraints, notamment en zones rurales, et s’avère particulièrement pertinente pour des chaînes de valeur agroalimentaires où la flexibilité et l’efficacité sont essentielles \cite{tastan2023}.

Les travaux récents en vision par ordinateur appliquée à l’agriculture illustrent déjà ce potentiel : détection précoce de maladies, évaluation de la maturité des fruits ou suivi de la qualité de denrées périssables. Appliquer ces avancées à la prédiction du temps de cuisson des haricots s’inscrit donc dans une dynamique à la fois scientifique et socio-économique. Il s’agit non seulement d’améliorer la maîtrise des procédés industriels, mais aussi de contribuer à la sécurité alimentaire mondiale face aux défis démographiques à venir, en proposant des solutions accessibles, économes en énergie et adaptées aux réalités des producteurs.

\section{Problématique}
Malgré les avancées technologiques, la prédiction précise et non destructive du temps de cuisson des légumineuses demeure un défi. Les approches traditionnelles présentent plusieurs limites :
\begin{itemize}
    \item \textbf{Temps et coût} : des méthodes chronophages, peu adaptées à un contrôle en temps réel ou à grande échelle.
    \item \textbf{Manque de portabilité} : des dispositifs lourds et onéreux, difficilement accessibles aux petites unités.
    \item \textbf{Contraintes matérielles et énergétiques} : peu de solutions compatibles avec les ressources limitées des environnements ruraux ou des micro-entreprises.
\end{itemize}

La question centrale à laquelle ce travail tente de répondre est donc la suivante :
\begin{quote}
    \textit{Comment concevoir et déployer un système de prédiction du temps de cuisson des haricots, fondé sur l’analyse d’images, qui soit à la fois précis, portable et compatible avec les contraintes matérielles et énergétiques du TinyML ?}
\end{quote}

\section{Objectifs et contributions}

\subsection{Objectif général}
Développer et valider un modèle TinyML capable de prédire le temps de cuisson des haricots à partir d’images, de manière non destructive, fiable et en quasi temps réel.

\subsection{Objectifs spécifiques}
\begin{enumerate}
    \item Constituer et prétraiter un jeu de données d’images de haricots annotées par leur temps de cuisson.
    \item Concevoir, entraîner et optimiser des modèles légers adaptés aux contraintes du TinyML (CNN compacts, MobileNetV2, EfficientNet-lite, etc.).
    \item Évaluer les performances des modèles selon des critères de précision, de consommation mémoire et de temps d’inférence.
    \item Intégrer et tester le modèle final sur une plateforme embarquée (smartphone).
\end{enumerate}

\subsection{Contributions attendues}
\begin{itemize}
    \item Une méthodologie reproductible et généralisable pour la prédiction visuelle du temps de cuisson des légumineuses.
    \item Un modèle optimisé, validé expérimentalement et conforme aux contraintes du TinyML.
    \item Une démonstration fonctionnelle sur dispositif embarqué, attestant de la faisabilité et de l’impact potentiel de l’approche en contexte réel.
\end{itemize}

\section{Organisation du mémoire}
Le mémoire est structuré de manière à guider le lecteur du cadre conceptuel aux résultats pratiques :
\begin{itemize}
    \item \textbf{Abstract} : synthèse du sujet, de la méthodologie et des résultats principaux.
    \item \textbf{Chapitre 1 — Introduction générale} : mise en contexte, problématique, objectifs et contributions.
    \item \textbf{Chapitre 2 — Revue de la littérature} : analyse critique des travaux antérieurs sur la prédiction du temps de cuisson, le TinyML et la vision par ordinateur appliquée à l’agroalimentaire.
    \item \textbf{Chapitre 3 — Méthodologie et conception} : description de la démarche expérimentale et des choix de conception.
    \item \textbf{Chapitre 4 — Implémentation et déploiement} : présentation des choix technologiques, de l’implémentation et du déploiement embarqué.
    \item \textbf{Chapitre 5 — Résultats et discussion} : analyse des performances obtenues et comparaison avec l’état de l’art.
    \item \textbf{Chapitre 6 — Conclusion et perspectives} : synthèse des contributions et identification de pistes futures.
\end{itemize}
